\chapter*{Introduction}
\addcontentsline{toc}{chapter}{Introduction}


L'objectif de ce projet est de comprendre, maîtriser, et utiliser la technologie des cycleGAN, proposée par Zhu et al. \cite{zhu_unpaired_2018} en 2017. Un cycleGAN est un algorithme particulier de traitement des images, qui prend la forme d'un réseau de neurones. La problématique à laquelle répond le cycleGAN est celle du transfert de style, cela signifie que l'on cherche à travailler une donnée structurée pour en modifier l’apparence globale. Par exemple, la transformation des objets d'une image, le changement de style pictural ainsi que le changement de style musical sont des transferts de style, et peuvent être abordés par l'utilisation d'un cycleGAN. C'est un problème particulièrement difficile pour un algorithme, en particulier lorsque les données ne sont pas appairées d'un style à un autre.\\

Le projet s'articule autour des cycleGAN, cependant ceux-ci reposent grandement sur la famille d'algorithmes des GAN introduite par Ian Goodfellow \cite{goodfellow_generative_2014-1}, qui reposent eux-même sur beaucoup d'autres concepts de \textit{machine learning}. C'est pourquoi, pour comprendre les cycleGAN, nous devons d'abord passer par plusieurs autres étapes importantes. Nous poserons d'abord les bases du \textit{machine learning}, en partant du simple \textbf{perceptron multicouches (Chapitre 1)}. Ensuite nous étudierons la spécificité des \textbf{couches à convolutions, ou 
CNN (Chapitre 2)} indispensables au traitement des données structurées compositionnelles telles que les images, et développées notamment par Yann LeCun \cite{lecun_gradient-based_1998}. Puis nous nous intéresserons aux \textbf{GAN ou Réseaux Adverses Génératifs (Chapitre 3)}. Il existe plusieurs architectures de GAN, nous en verrons les deux types principaux : les DCGAN proposés par Radford et al. en 2016 \cite{radford_unsupervised_2016} et les W-GAN proposés par Arjovsky et al. en 2017 \cite{arjovsky_wasserstein_2017}. Enfin, grâce à tous ces outils, nous pourrons comprendre le fonctionnement des \textbf{cycleGAN (Chapitre 4)}. Nous chercherons ensuite à appliquer notre cycleGAN au remplissage d'images (image inpainting). L'idée de base est de retirer par exemple un individu d'une photo. Pour ceci, nous étudierons le \textbf{MaskRCNN (Chapitre 5)} permettant d'identifier dans l'image la zone à retirer. Tous ces points sont accompagnés de l'implémentation des algorithmes sur TensorFlow 2.0 \cite{goldsborough_tour_2016}.\\

Ce rapport constitue un résumé des connaissances que nous avons acquises.
